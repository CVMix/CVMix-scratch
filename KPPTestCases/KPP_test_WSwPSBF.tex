\chapter{Wind-stress with positive surface buoyancy forcing
   (WSwPSBF)}
\label{chapter:WSwPSBF}

 These tests are configured with a wind stress providing mechanical
 mixing that is countered by generally non-negative surface buoyancy
 forcing realized by a heat flux.  Since the buoyancy forcing is
 generally positive (stabilizing), these tests only exercise the local
 downgradient diffusive portion of KPP.  In this way, the tests are
 very simple.  In turn, the tests provide a useful means to ensure
 that implementation details are consistent across models making use
 of CVMix/KPP.


\section{Physics of the experiments}

There are no lateral gradients in these column-tests.  Hence,
evolution of velocity and tracer are one-dimensional in the vertical.
We generally start the velocity field at zero, so that its evolution
is initiated by surface stresses.  Likewise, if tracers have a uniform
initial profile, their evolution is initiated by surface heat or
salt/water fluxes.  Once a nontrivial vertical profile is established,
either through boundary fluxes or initial profiles, then evolution is
impacted by local (downgradient diffusion) and non-local aspects of
KPP.

In these test cases, buoyancy is forced at the surface via a surface
heat flux.  To help develop notation, we express the finite-volume
heat equation acting on the top grid cell as
\begin{equation}
  \mbox{C}_{\mbox{\footnotesize p}} \left(   \frac{\partial  \, (\theta \, \rho \, \mathrm{d}z )}{\partial t}  \right) =  
  Q + F_{\mbox{\footnotesize diffusive}} + F_{\mbox{\footnotesize non-local}}.
\label{eq:surface-heat-equation}
\end{equation}
The following terms appear in this equation.
\begin{itemize}
\item The grid cell thickness, $\mathrm{d}z$, is generally a function
  of time, as in MPAS and MOM, in which case we keep it inside the
  time derivative to ensure a flux-form budget.  If the cell thickness
  is static, as in POP, then it can be removed from the time
  derivative.

\item The {\it in situ} density factor, $\rho$, is replaced by the
  constant reference density $\rho_{0}$ when making the Boussinesq
  approximation.

\item The surface boundary heat flux, $Q$, crosses the top of the
  surface cell.  It is specified in these tests according to the
  details in Table \ref{table:kpp-model-configuration} discussed in
  Section \ref{section:configuration-details-WSwPSBF}.

\item The diffusive heat flux, $F_{\mbox{\footnotesize diffusive}}$,
  arises from the local downgradient diffusive portion of the KPP
  scheme.  This flux passes through the bottom face of the top grid
  cell.

\item The non-local KPP term, $F_{\mbox{\footnotesize non-local}}$,
  vanishes in these tests since we are only considering positive
  buoyancy forcing.

\end{itemize}


\section{Configuration details}
\label{section:configuration-details-WSwPSBF}

We here provide configuration details.  Note that each ocean model
makes the Boussinesq approximation for these tests.


\subsection{Vertical grids and time steps}

We consider three configurations for each ocean model (MOM6, MPAS, and
POP).  Two tests are identical across the models, based on uniform
vertical grids of $\Delta z = 1~\mbox{m}$ and $\Delta z =
10~\mbox{m}$.  These tests allow us to examine differences arising
from coarsening the vertical resolution.  The third test is unique to
each ocean model, and is based on a global configuration that uses
non-uniform grid spacing.

%%%%%%%%%%%%%%%%%%%% model configurations%%%%%%%%%%%%%%%%%%%%%%%%
\begin{table}[h!t]
\centering{
\begin{tabular}{|c|c|c|c|}
\hline 
{\sc model}                        & {\sc vertical grid spacing/grid points}    & {\sc bottom depth}   &    {\sc time step (secs)} \\
\hline uniform-1                 & $\Delta z = 1~\mbox{m}$                         & $400~\mbox{m}$      & $\Delta t = 20 * 60$     \\                               
\hline uniform-10               & $\Delta z = 10~\mbox{m}$                       & $400~\mbox{m}$      & $\Delta t = 20 * 60$      \\                               
\hline MOM6-CM4             &  75                                                           & $6000~\mbox{m}$    & $\Delta t = 30 * 60$      \\
\hline MPAS-ACME             &  ??                                                             & ??                            & $\Delta t = ??$             \\
\hline POP-CESM                & ??                                                            & ??                              & $\Delta t = ??$             \\
\hline
\end{tabular}
}
\caption{Model vertical grid spacing, bottom depth,  and time steps for the CVMix column tests.}
\label{table:kpp-model-configuration}
\end{table}
%%%%%%%%%%%%%%%%%%%%%%%%%%%%%%%%%%%%%%%%%%%%%%%%%%%%%%%%%%%%%%%%%%%%%%%%


\subsection{Basic constants and linear equation of state}

The basic constants used for the tests are given by the gravitational
acceleration, the Coriolis parameter, and the heat capacity, each of
which are global constants.
\begin{subequations}
\begin{align}
 g &= 9.80616~\mbox{m}~\mbox{s}^{-2}
 \\
 f &= 1.0 \times 10^{-4}~\mbox{s}^{-1}
 \\
\mbox{C}_{\mbox{\footnotesize p}} &= 3992~\mbox{J}~\mbox{kg}^{-1}~\mbox{K}^{-1}. 
\end{align}
\end{subequations}
We offer two comments here.
\begin{itemize}
 \item A non-zero Coriolis parameter allows for the introduction of
inertial oscillations as part of the vertical column test cases.
Indeed, we generally see inertial oscillations as a prominent feature
in these tests.   
\item The heat capacity is needed to convert from a heat flux to a
  temperature flux.  The chosen value accords with four significant
  digits of the \cite{TEOS2010} value.

\end{itemize}

The equation of state for density is linear, and a function only of
temperature and salinity 
\begin{equation}
\rho = \rho_0 - \alpha \, (\theta - \theta_0) + \beta \, (S-S_0).
\label{eq:linear-eos}
\end{equation}
The equation of state constants are given by 
\begin{subequations}
\begin{align}
 \rho_0 &= 1025.022~\mbox{kg}~\mbox{m}^{-3}
\label{eq:linear-eos-paramA}
 \\
\alpha &= 2.55 \times 10^{-1}~\mbox{kg}~\mbox{m}^{-3}~^{\circ}C^{-1}
\\
\beta &= 7.64\times10^{-1}~\mbox{kg}~\mbox{m}^{-3}~\mbox{ppt}^{-1}
 \\
T_0  &=  19^{\circ}C 
\\
S_0 &=  35~\mbox{ppt}.
\label{eq:linear-eos-paramB}
\end{align}
\end{subequations}


\subsection{Initial Conditions}

We start the velocity field with zero value throughout the column
\begin{equation}
 {\bf v} = 0.
\end{equation}
Surface stress will impose a vertical gradient in velocity, on which
the KPP viscosity will transfer momentum vertically.  Salinity is
initialized to the constant value
\begin{equation}
 S = S_{0}.
\end{equation}
There are no salt fluxes through the boundaries, so that salinity
should remain constant throughout these tests.  The temperature is
initialized with a linear vertical stratification
\begin{subequations}
\begin{align}
 \theta(z) &= \theta_s + \Gamma \, z  
\\
 \theta_s &= 15.0^{\circ}C 
\\
 \Gamma  &= 0.01^{\circ}C~\mbox{m}^{-1}. 
\end{align}
\end{subequations}
Even without a surface buoyancy flux, temperature will evolve in the
presence of a nonzero diffusivity.  With the linear equation of state
(\ref{eq:linear-eos}) and parameters
(\ref{eq:linear-eos-paramA})-(\ref{eq:linear-eos-paramB}), the initial
vertical temperature stratification leads to an initial buoyancy
frequency
\begin{subequations}
\begin{align}
 N &= \left(-\frac{g}{\rho_{0}} \frac{\partial \rho}{\partial z} \right)^{1/2}
 \\
   &= \left( \frac{g \, \alpha \, \Gamma}{\rho_{0}} \right)^{1/2}
\\
  &= 4.94 \, \times \, 10^{-3}~\mbox{s}^{-1}
 \\
 &= 49.4 \, f.
\end{align}
\end{subequations}


\subsection{Boundary conditions}

We assume the following boundary conditions.
\begin{itemize}

\item Zero buoyancy flux passes through the ocean bottom.

\item Constant zonal wind stress forces the surface ocean, $\tau^{x} =
  0.1~\mbox{N}~\mbox{m}^{-2}$, along with a zero meridional stress,
  $\tau^{y} = 0$.

\item Specification of the surface heat flux, $Q$ (see equation
  (\ref{eq:surface-heat-equation})), distinguishes the following three
  experiments (see Table \ref{table:kpp-test-case-WSwPSBF}).

\begin{itemize} 

\item {\sc WSwPSBF.A} uses zero surface heat flux, $Q=0$.

\item {\sc WSwPSBF.B} uses a constant surface heat flux,
  $Q=100~\mbox{W}~\mbox{m}^{-2}$.

\item {\sc WSwPSBF.C} uses a restoring surface boundary condition
  given by
\begin{equation}
 Q =  \rho_o \, \mbox{C}_{\mbox{\footnotesize p}} \, w_{\mbox{\footnotesize piston}} \, (\theta - \theta_{\mbox{\footnotesize restore}} ),
\end{equation}
 where the restoring temperature is 
\begin{equation}
\theta_{\mbox{\footnotesize restore}}  = \theta_s + 10^{\circ}C,
\end{equation}
 and the piston velocity is 
\begin{equation}
  w_{\mbox{\footnotesize piston}} = 0.5~\mbox{m}~\mbox{day}^{-1}. 
\end{equation}
With these parameters, the restoring boundary condition produces a
heat flux, per degree temperature deviation $\theta -
\theta_{\mbox{\footnotesize restore}}$, given by
\begin{subequations}
\begin{align}
 \frac{Q}{(\theta - \theta_{\mbox{\footnotesize restore}} )} &= \rho_o \, \mbox{C}_{\mbox{\footnotesize p}} \, w_{\mbox{\footnotesize piston}}
 \\
 &= 23.68~\mbox{W}~\mbox{m}^{-2}~\mbox{K}^{-1}.  
\end{align}
\end{subequations}
That is, with $\theta - \theta_{\mbox{\footnotesize restore}} =
1~\mbox{K}$, we have $Q = 23.68~\mbox{W}~\mbox{m}^{-2}$.  This heat
flux will always be non-negative, causing the ocean temperature to
initially increase quite rapidly as driven by
$\theta_{\mbox{\footnotesize restore}} - \theta_s = 10^{\circ}C$, and
with a decreasing (still non-negative) heat flux as the surface
temperature approaches $\theta_{\mbox{\footnotesize restore}}$.

\end{itemize} 

\end{itemize}


%%%%%%%%%%%%%%%%%%%% boundary heat fluxes%%%%%%%%%%%%%%%%%%%%%%%
\begin{table*}[h!t]
\centering{
\begin{tabular}{|c|c|}
\hline 
{\sc test case name}             & {\sc heat flux} ($Q$)      \\
\hline {\sc WSwPSBF.A}          & 0                                                                                  \\
\hline {\sc WSwPSBF.B}          & $Q = 100~\mbox{W}~\mbox{m}^{-2}$                        \\
\hline {\sc WSwPSBF.C}           & $Q = \rho_o \, \mbox{C}_{\mbox{\footnotesize p}} \, w_{\mbox{\footnotesize piston}} \, (\theta - \theta_{\mbox{\footnotesize restore}} )$   \\
\hline
\end{tabular}
}
\caption{
  Surface boundary heat flux formulations for the three test cases.  
  The third test considers a restoring boundary condition
  for the top model grid cell, with restoring temperature given by  
  $\theta_{\mbox{\footnotesize restore}} 
  = \theta_s + 10^{\circ}C$, and the piston velocity set to 
  $w_{\mbox{\footnotesize piston}} = 0.5~\mbox{m}~\mbox{day}^{-1}$.}
\label{table:kpp-test-case-WSwPSBF}
\end{table*}
%%%%%%%%%%%%%%%%%%%%%%%%%%%%%%%%%%%%%%%%%%%%%%%%%%%%%%%%%%%%%%%%%%%%%%%%


\subsection{CVMix/KPP parameter settings}

The following KPP parameters are set for these tests. 
\begin{itemize}

\item Critical Richardson number is set to the standard value $\mbox{Ri}_{\mbox{\footnotesize c}} = 0.3$.

\item The background viscosity and diffusivity are both set to zero,
  so that the only means of diffusing tracer or velocity arises from
  the KPP diffusivity.

\item The KPP boundary layer interpolation type is set to {\tt cubic}.

\item The surface layer thickness occupies 10\% of the KPP boundary
  layer depth.  This thickness generally spans more than just the
  surface grid cell, especially when $\Delta z = 1~\mbox{m}$.  If one
  instead assumes the surface layer is just the surface grid cell,
  then simulation results will differ from those presented here.

\item The buoyancy forcing is positive, so the non-local transport
  vanishes.  Hence, there is no need to detail associated choices for
  the non-local transport.

\end{itemize}


\section{Evaluation diagnostics}

We aim to evaluate the simulation in various ways, both to expose the
physical processes and to allow for detailed comparison across a suite
of models implementing the CVMix/KPP scheme.  We list certain of the
diagnostics in Table \ref{table:metricsShear} of use for this purpose.
Ideally, all models will produce the same results.

\begin{table}[htdp]
\begin{center}
\begin{tabular}{|c|c|c|c|c|}
\hline
{\sc quantity}                                 & {\sc name}        & {\sc units}                              & {\sc output sampling}     & {\sc comments} \\
\hline
surface boundary-layer depth          & $OBL$              & $\mbox{m}$                            & $\Delta t$                       &  \\
\hline 
sea-surface temperature                 & $SST$               & $^{\circ}C$                                & $\Delta t$                       & $SST=T(k=1)$\\
\hline 
surface buoyancy flux                     & $B_{f}$              & $\mbox{m}^2~\mbox{s}^{-3}$   &$\Delta t$                        & \\
\hline 
position of layer interfaces              & $z_I$                & $\mbox{m}$                             & $\Delta t$ &  \\
\hline 
zonal velocity component               & $u$                   & $\mbox{m}~\mbox{s}^{-1}$       &  $\Delta t$  &  all vertical levels \\
\hline 
meridional velocity component       & $v$                    & $\mbox{m}~\mbox{s}^{-1}$       &  $\Delta t$  &  all vertical levels \\
\hline
\end{tabular}
\end{center}
\caption{The first output should be at the initial condition, $t=0$. The final 
  output should be at the end of the test, at $t=15~\mbox{days}$.  All vertical grid points should be sampled.}
\label{table:metricsShear}
\end{table}

\section{Results}
\subsection{LANL MPAS-Ocean}
\subsection{NCAR POP}
\subsection{GFDL MOM6}



