\chapter{Common elements to the tests}
\label{chapter:common_elements}

In this chapter, we document common elements for the test cases.

\minitoc


\section{Heat equation and notation}

In these test cases, buoyancy is forced via a surface heat flux, which
is either zero, positive, or negative.  To help develop notation,
consider the finite-volume one-dimensional heat budget for a surface
ocean grid cell
\begin{equation}
  \mbox{C}_{\mbox{\footnotesize p}} \left(   \frac{\partial  \, (\Theta \, \rho \, \mathrm{d}z )}{\partial t}  \right) =  
  Q + F_{\mbox{\footnotesize diffusive}} + F_{\mbox{\footnotesize non-local}}.
\label{eq:surface-heat-equation}
\end{equation}
All lateral transport terms are dropped given that these tests
exercise just the vertical transport.  The following terms appear in
this equation.
\begin{itemize}

\item Conservative temperature, $\Theta$, is approximated in these
  tests with potential temperature, $\theta$.  Indeed, since the
  equation of state is linear and without pressure dependence (Section
  \ref{subsection:equation-of-state}), there is no distinction between
  Conservative, potential, or {\it in situ} temperatures.

\item The mass per unit area of a grid cell is written as $\rho \,
  \mathrm{d}z$.  For a Boussinesq model, $\rho$ in the tracer budget
  is set to the constant reference density $\rho_{0}$.  Each ocean
  model makes the Boussinesq approximation for the tests documented
  here.  The grid cell thickness, $\mathrm{d}z$, is generally a
  function of time, as in MPAS-ocean and MOM6, in which case we keep
  it inside the time derivative to ensure a flux-form budget.  If the
  cell thickness is static, as in POP, then it can be removed from the
  time derivative.\footnote{Note that for hydrostatic non-Boussinesq
    fluids, $\rho \, \mathrm{d}z = -g^{-1} \, \mathrm{d}p$, measures
    the mass per unit horizontal area for the grid cell, with
    $\mathrm{d}p$ the pressure increment for the cell.}

\item The surface boundary heat flux, $Q$, crosses the top of the
  surface cell.  The different test cases use various heat heat flux
  specifications.

\item The diffusive heat flux, $F_{\mbox{\footnotesize diffusive}}$,
  arises from the local downgradient diffusive portion of the KPP
  scheme.  This flux passes through the bottom face of the top grid
  cell.

\item The non-local KPP term, $F_{\mbox{\footnotesize non-local}}$, is
  non-zero only when the surface buoyancy flux is negative.  The
  non-local term acts to redistribute the negative surface buoyancy
  flux throughout the boundary layer.  The non-local term vanishes at
  the base of the boundary layer.

\end{itemize}




\section{Configuration details}
\label{section:configuration-details-winds_alone}

We provide configuration details in this section.  


\subsection{Vertical grids and time steps}

As detailed in Table \ref{table:kpp-model-configuration}, we consider
various vertical grid configurations for each ocean model (MOM6,
MPAS-ocean, and POP). These tests serve as a platform to examine
differences arising from coarsening the vertical resolution.

%%%%%%%%%%%%%%%%%%%% model configurations%%%%%%%%%%%%%%%%%%%%%%%%
\begin{table}[h!t]
\centering{
\begin{tabular}{|c|c|c|c|}
\hline 
{\sc model}                        & {\sc vertical grid spacing/grid points}    & {\sc bottom depth}   &    {\sc time step (secs)} \\
\hline 0.4m              & $\Delta z = 0.4~\mbox{m}$                     & $400~\mbox{m}$      & $\Delta t = 20 * 60$     \\                               
\hline 1m                 & $\Delta z = 1.0~\mbox{m}$                      & $400~\mbox{m}$      & $\Delta t = 20 * 60$     \\                               
\hline 10m               & $\Delta z = 10.0~\mbox{m}$                    & $400~\mbox{m}$      & $\Delta t = 20 * 60$      \\                               
\hline MOM6-CM4             &  75                                                   & $6500~\mbox{m}$    & $\Delta t = 20 * 60$      \\
%\hline MPAS-ACME             &  ??                                                             & ??                            & $\Delta t = ??$             \\
%\hline POP-CESM                & ??                                                            & ??                              & $\Delta t = ??$             \\
\hline
\end{tabular}
}
\caption{Model vertical grid spacing, bottom depth,  and time steps
  for the CVMix column tests.  Note that each test 
  with MOM6 was run on a workstation, which constrains the size of the
  finest grid spacing.   The MOM6-CM4 vertical spacing corresponds to
  the spacing used in the CM4 global climate model being developed 
  at GFDL.}
\label{table:kpp-model-configuration}
\end{table}
%%%%%%%%%%%%%%%%%%%%%%%%%%%%%%%%%%%%%%%%%%%%%%%%%%%%%%%%%%%%%%%%%%%%%%%%


\subsection{Basic constants and linear equation of state}
\label{subsection:equation-of-state}

The basic constants used for the tests are given by the gravitational
acceleration, the Coriolis parameter, and the heat capacity, each of
which are global constants.
\begin{subequations}
\begin{align}
 g &= 9.80616~\mbox{m}~\mbox{s}^{-2}
 \\
 f &= 1.0 \times 10^{-4}~\mbox{s}^{-1}
 \\
\mbox{C}_{\mbox{\footnotesize p}} &= 3992.1~\mbox{J}~\mbox{kg}^{-1}~\mbox{K}^{-1}. 
\end{align}
\end{subequations}
We offer the following comments.
\begin{itemize}
\item A non-zero Coriolis parameter allows for inertial oscillations
  in the vertical column test cases, with periods 
\begin{equation}
   T_{\mbox{\footnotesize inertial}} = \frac{2 \, \pi}{f} = 0.73~\mbox{day}.
\label{eq:inertial-period}
\end{equation}
Consequently, inertial oscillations are a prominent feature in the
velocity field realized in these tests.

\item The heat capacity is needed to convert from a heat flux to a
  temperature flux.  The chosen value accords with the first five
  significant digits of the \cite{TEOS2010} value.

\end{itemize}
The equation of state for density is linear, and a function only of
temperature and salinity 
\begin{equation}
\rho = \rho_0 - \alpha \, (\Theta - \Theta_0) + \beta \, (S-S_0).
\label{eq:linear-eos}
\end{equation}
The equation of state constants are given by 
\begin{subequations}
\begin{align}
 \rho_0 &= 1000.00~\mbox{kg}~\mbox{m}^{-3}
\label{eq:linear-eos-paramA}
 \\
\alpha &= 2.55 \times 10^{-1}~\mbox{kg}~\mbox{m}^{-3}~^{\circ}C^{-1}
\\
\beta &= 7.64\times10^{-1}~\mbox{kg}~\mbox{m}^{-3}~\mbox{ppt}^{-1}
 \\
T_0  &=  19^{\circ}C 
\\
S_0 &=  35~\mbox{ppt}.
\label{eq:linear-eos-paramB}
\end{align}
\end{subequations}
  Given that the equation of state is linear, there is no distinction
  between Conservative temperature, potential temperature, or {\it in
    situ} temperature.  



\subsection{Initial Conditions}

We start the velocity field with zero value throughout the column
\begin{equation}
 {\bf v} = 0.
\end{equation}
Surface stress imparts a vertical shear in the horizontal velocity.  A
vertical viscosity will then transfer momentum vertically.  Salinity
is initialized to the constant value
\begin{equation}
 S = S_{0}.
\end{equation}
There are no salt fluxes through the boundaries, so that salinity
remains constant throughout these tests.  Checking that it indeed
remains constant is a useful means to test that the model code is
properly conserving the salt content, both locally and globally
\citep{GriffiesPacSchmidtBalaji2001}.  The temperature is initialized
either as a uniform constant of $ \Theta_s = 15.0^{\circ}\mbox{C}$
(winds plus heating test in Chapter \ref{chapter:winds_and_heating}),
or with a linear vertical stratification
\begin{subequations}
\begin{align}
 \Theta(z) &= \Theta_s + \Gamma \, z  
\label{eq:linear-stratification}
\\
 \Theta_s&= 15.0^{\circ}\mbox{C} 
\label{eq:thetas}
\\
 \Gamma  &= 0.01^{\circ}\mbox{C}~\mbox{m}^{-1}. 
\end{align}
\end{subequations}
Even without a surface buoyancy flux, temperature will evolve in the
presence of a nonzero vertical diffusivity.  With the linear equation
of state (\ref{eq:linear-eos}) and parameters
(\ref{eq:linear-eos-paramA})-(\ref{eq:linear-eos-paramB}), the initial
vertical temperature stratification leads to an initial buoyancy
frequency
\begin{subequations}
\begin{align}
 N &= \left(-\frac{g}{\rho_{0}} \frac{\partial \rho}{\partial z} \right)^{1/2}
 \\
   &= \left( \frac{g \, \alpha \, \Gamma}{\rho_{0}} \right)^{1/2}
\\
  &= 4.94 \, \times \, 10^{-3}~\mbox{s}^{-1}
 \\
 &= 49.4 \, f.
\end{align}
\end{subequations}
We are movitated to initialize the heating case with uniform
temperature in order to record the development of nontrivial
stratification under heating. 


\subsection{Boundary conditions}

We assume the following boundary conditions.
\begin{itemize}

\item We assume a constant zonal stress and zero meridional stress at
  the ocean surface
\begin{subequations}
\begin{align}
 \tau^{x} &=  0.1~\mbox{N}~\mbox{m}^{-2}
 \\
 \tau^{y} &=  0.
\end{align} 
\end{subequations}

\item Zero buoyancy flux passes through the ocean bottom.

\item Constant surface boundary heat flux, either zero, positive, or
  negative.

\item We considered a test case with a restoring surface boundary
  condition.  The results were found to be quite similar to the
  constant positive heat case.  Hence, the restoring test case was
  dropped in order to streamline testing. However, it may prove of use
  in future tests, which motivates providing the following details.

  The restoring heat flux given by
\begin{equation}
 Q =  \rho_o \, \mbox{C}_{\mbox{\footnotesize p}} \, w_{\mbox{\footnotesize piston}} \, (\Theta_{\mbox{\footnotesize restore}}  - \Theta),
\label{eq:heat-flux-restoring}
\end{equation}
 where the restoring temperature is 
\begin{equation}
\Theta_{\mbox{\footnotesize restore}}  = \Theta_s + 10^{\circ}C,
\label{eq:restoring-temp}
\end{equation}
 and the piston velocity is 
\begin{equation}
  w_{\mbox{\footnotesize piston}} = 0.5~\mbox{m}~\mbox{day}^{-1}. 
\end{equation}
With these parameters, the restoring boundary condition produces a
heat flux, per degree temperature deviation
$\Theta_{\mbox{\footnotesize restore}} - \Theta$, given by
\begin{subequations}
\begin{align}
 \frac{Q}{(\Theta_{\mbox{\footnotesize restore}} - \Theta )} &= \rho_o \, \mbox{C}_{\mbox{\footnotesize p}} \, w_{\mbox{\footnotesize piston}}
 \\
 &= 23.68~\mbox{W}~\mbox{m}^{-2}~\mbox{K}^{-1}.  
\end{align}
\end{subequations}
That is, with $\Theta_{\mbox{\footnotesize restore}} - \Theta =
1~\mbox{K}$, we have $Q = 23.68~\mbox{W}~\mbox{m}^{-2}$.  This heat
flux is always non-negative, so it causes the ocean temperature to
initially increase quite rapidly as driven by
$\Theta_{\mbox{\footnotesize restore}} - \Theta_s = 10^{\circ}C$, and
with a decreasing (still non-negative) heat flux as the surface
temperature approaches $\Theta_{\mbox{\footnotesize restore}}$.  

\end{itemize}



\subsection{CVMix/KPP parameter settings}

The following KPP parameters are set for these tests. 
\begin{itemize}

\item Critical Richardson number is set to the standard value $\mbox{Ri}_{\mbox{\footnotesize c}} = 0.3$.

\item The background viscosity and diffusivity are both set to zero,
  so that the only means of diffusing tracer or velocity arises from
  the KPP diffusivity.

\item The KPP boundary layer interpolation type is set to {\tt cubic}.

\item The surface layer thickness occupies 10\% of the KPP boundary
  layer depth.  This thickness generally spans more than just the
  surface grid cell.  If one instead assumes the surface layer is just
  the surface grid cell (not recommended), then simulation results
  will differ from those presented here.

\item The non-local term is treated via the {\tt SimpleShapes} option.

\end{itemize}


\section{Diagnostics}

We aim to evaluate the simulation in various ways, both to expose the
physical processes and to allow for detailed comparison across a suite
of models implementing the CVMix/KPP scheme.  We list certain of the
diagnostics in Table \ref{table:metrics} of use for this purpose.
Ideally, all models will produce the same results.  However, we have
found that results can depend on the differences in time stepping
schemes used by the models (leap-frog with periodic Euler backward in
POP; forward-backward in MOM6; and Runga-Kutta in MPAS-O).  

\begin{table}[htdp]
\begin{center}
\begin{tabular}{|c|c|c|c|c|}
\hline
{\sc quantity}                                 & {\sc name}        & {\sc units}                              & {\sc output sampling}     & {\sc comments} \\
\hline
surface boundary-layer depth          & $OBL$              & $\mbox{m}$                            & $\Delta t$                       &  \\
\hline 
mixed layer depth                           & $MLD$              & $\mbox{m}$                            & $\Delta t$                       &  \\
\hline 
sea-surface temperature                 & $SST$               & $^{\circ}C$                                & $\Delta t$                       & $SST=\Theta(k=1)$\\
\hline 
% surface buoyancy flux                     & $B_{f}$              & $\mbox{m}^2~\mbox{s}^{-3}$   &$\Delta t$                    &  \\
% \hline 
% position of layer interfaces              & $z_I$                & $\mbox{m}$                             & $\Delta t$ &  \\
% \hline 
temperature                                    & $\Theta$         & $^{\circ}C$                                  &  $\Delta t$  &  all vertical levels \\
\hline
 zonal velocity component               & $u$                   & $\mbox{m}~\mbox{s}^{-1}$       &  $\Delta t$  &  all vertical levels \\
\hline 
%meridional velocity component       & $v$                    & $\mbox{m}~\mbox{s}^{-1}$       &  $\Delta t$  &  all vertical levels \\
%\hline
\end{tabular}
\end{center}
\caption{Table of diagnostics for the various test cases. 
  The first output should be at the initial condition, $t=0$. The final 
  output is at the end of the test, at $t=15~\mbox{days}$.  
  All vertical grid points are sampled, and all time steps are
  sampled. SST is diagnosed at the temperature in the top model grid
  cell.  The mixed layer depth is diagnosed as the depth where the
  potential density referenced to the surface differs from the surface
  potential density by $0.003~\mbox{kg}~\mbox{m}^{-3}$.  The mixed layer
  depth does not equal to the boundary layer depth, so it is useful to
  diagnose both.}
\label{table:metrics}
\end{table}



