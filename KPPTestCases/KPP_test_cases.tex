%\documentclass[11pt][fleqn]{report}
\documentclass[fleqn, 12pt]{report}

\usepackage{epsf,amsmath,amsfonts}
\usepackage{graphicx}
\usepackage{parskip}

\setlength{\textwidth}{6.5in}
\setlength{\oddsidemargin}{0in}
\setlength{\evensidemargin}{0in}
\setlength{\textheight}{8.5in}
\setlength{\topmargin}{0in}
\setlength{\parindent}{0cm}
\setlength{\mathindent}{0pt}


\begin{document}

\title{
Basic KPP test cases}
\author{T. Ringler, M. Levy, others}

\maketitle
\tableofcontents


\chapter{Summary}
This document is to specify a set of one-dimensional (column) test cases to evaluate mixed layer closures within CVMix. The test cases outlined here are very basic in the sense that forcing is time invariant.

\chapter{Wind-Stress with Positive Surface Buoyancy Forcing (WSwPSBF)}

\section{Constants}

$g=9.80616 \ m \ s^{-2}$.

$f=1.0 \times 10^{-4} \ s^{-1}$.

\section{CVMix Configuration}

Critical Richardson numbers set to $Ri_c = 0.3$

Background viscosity and diffusivity set to $\nu_b = 0~m^2 s^{-1}$.

KPP mixed layer interpolation type set to ``cubic''.

Question: Which CVMix mixing parameterization modules are turned on?

\section{Equation of State}
Linear EOS with
\begin{equation}
\rho = \rho_0 - \alpha (T-T_0) + \beta (S-S_0)
\end{equation}
where 
\begin{align*}
&\rho_0=1025.022 \ kg \ m^{-3} \\
&\alpha=2.55\times10^{-1} \ kg \ m^{-3} \ ^{\circ}C^{-1} \\
&\beta=7.64\times10^{-1} \ kg \ m^{-3} \ {PSU}^{-1} \\
&T_0= 19.0 \ ^{\circ}C \\
&S_0=35.0 \ PSU 
\end{align*}

\section{Initial Conditions}

\begin{align*}
&T(z) = T_s + \left( \frac{\delta T}{\delta z} \right) z  \\
&S(z) = S_0 \\
&{\bf u}(z) = 0
\end{align*}
where
\begin{align*}
&T_s=15.0 \ ^{\circ}C \\
&\frac{\delta T}{\delta z}=0.01 \ ^{\circ}C \ m^{-1} \\
&-400.0 \ m <z<0
\end{align*}
\section{Forcing}

Wind stress:
\begin{equation*}
{\bf \tau} = \tau_x {\bf i} + \tau_y {\bf j} \\
\end{equation*}
where
\begin{align*}
&\tau_x= 0.1 \ N \ m^{-2} \\
&\tau_y=0
\end{align*}

Three different applications of surface buoyancy forcing are defined:

WSwPSBF.A is no buoyancy forcing: $Q_T =0$

WSwPSBF.B is fixed temperature flux: $Q_T = 100.0$~W~m$^{-2}$

WSwPSBF.C is heat flux of the form $Q_T = \rho_o \, c_p \, w_p \left( T - T_{restore}\right)$ so that the top layer temperature is restored with the form $\partial_t T + \ldots = - \left( \frac{w_p}{\Delta z_1} \right) \left(T - T_{restore} \right)$

where $T_{restore}=T_s+10\ ^{\circ}C$ and $w_p = 0.5 \;m\;day^{-1}$.

Salinity remains constant throughout the simulation.

\section{Model Configuration}
\subsection{Reference Configuration}
 $0<t<15~days$

 $\Delta t = 20~mins$

 $\Delta z(k) = 1.0~m$ (uniform), so $1 \le k \le N=400$.
 

\subsection{Operational Configuration}
 $0<t<15 \ days$
 
 specify $dt=?$, $N=?$, $dz(k)=?$

\section{Evaluation Metrics}
see Table \ref{table:metricsShear}

\begin{table}[htdp]
\caption{First output at $t=0$. Last output at $t=15 \ days$, $1<k<N$, $kI=N+1$}
\begin{center}
\begin{tabular}{c c c c c}
\hline
Quantity & Name & Units & Output Freq & Comments \\
\hline
Boundary-layer depth & $OBL$ & $m$ & $dt$ &  \\
Sea-surface temperature & $SST$ & $^{\circ}C$ & $dt$ & $SST=T(k=1)$\\
Surface buoyancy flux & ? & $m^2 \ s^{-3}$ & dt & \\
Position of layer interfaces & $z_I$ & $m$ & $dt$ &  \\
Zonal velocity & $U_x$ & $m \ s^{-1}$ & $dt$ &  all vertical levels \\
Meridional velocity & $U_y$ & $m \ s^{-1}$ & $dt$ &  all vertical levels \\
Non-local KPP transport & $?$ & $??$ & $dt$ & all vertical levels \\
other & $other$ & $other$ & other & other\\
\hline
\end{tabular}
\end{center}
\label{table:metricsShear}
\end{table}

\section{Comments and Caveats}
Results will likely be sensitive to how surface layer is modeled.

\section{Results}
\subsection{LANL MPAS-Ocean}
\subsection{NCAR POP}
\subsection{GFDL MOM6}

\chapter{Wind-stress with negative surface buoyancy forcing}

\section{Initial Conditions}

\section{Forcing}

\section{Results}

%\bibliographystyle{}
%\bibliography{bibliography/master_bibliography}



\end{document}
